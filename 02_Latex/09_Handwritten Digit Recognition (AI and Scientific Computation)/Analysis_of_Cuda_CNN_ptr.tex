\subsubsection{代码分析比较}

对\colorbox[gray]{0.9}{\texttt{Cuda\_CNN\_ptr}}中的代码进行分析。并与\colorbox[gray]{0.9}{\texttt{Seq\_CNN\_course}}中的代码进行比较,\colorbox[gray]{0.9}{\texttt{Cuda\_CNN\_ptr}}程序主要由以下几个部分组成:

\begin{enumerate}
    \item \colorbox[gray]{0.9}{\texttt{drv\_cnn}}
    \item \colorbox[gray]{0.9}{\texttt{Activation\_layer}}
    \item \colorbox[gray]{0.9}{\texttt{convolutional\_layer}}
    \item \colorbox[gray]{0.9}{\texttt{pooling\_layer}}
    \item \colorbox[gray]{0.9}{\texttt{fully\_connected\_layer}}
    \item \colorbox[gray]{0.9}{\texttt{softmax\_layer}}
\end{enumerate}

此程序主页也是由此六大程序组成,分别为顶层程序、激活层、卷积层、池化层、全连接层和softmax层组成,其结构与\colorbox[gray]{0.9}{\texttt{Seq\_CNN\_course}}基本相同,可以认为结构基本一样,只是在每一层的具体实现上,利用了CUDA编程,利用GPU并行加速程序运行时间。
